%%%%%%
%%
%%  Don't reorder the reviewer points; that'll mess up the automatic referencing!
%%
%%%%%

\begin{minipage}[b]{2.5in}
  Resubmission Cover Letter \\
    {\it G3}
\end{minipage}
\hfill
\begin{minipage}[b]{2.5in}
    Jared Galloway, \\
    William Cresko, \\
    \emph{and} Peter Ralph \\
  \today
\end{minipage}
 
\vskip 2em
 
\noindent
{\bf To the Editor(s) -- }
 
\vskip 1em

Thanks very much for the prompt and positive decision.
We have added some additional discussion about the two reviewer points
(that previously we had only responded to in the Response to Reviewers),
as requested by the editor.
This was a good idea, as other readers will likely have the same thoughts.
One of the points turned out to be somewhat tricky to discuss in an accurate way,
since it involved speculating beyond the bounds of our study,
but we've got something in that we're satisified with.

We hope that you agree with us.

In addition to the revised manuscript (Stickleback\_Paper\_text.pdf),
we have uploaded a version with the changes we have made highlighted (Stickleback\_Paper\_diff.pdf).


\vspace{5em}

\noindent \hspace{4em}
\begin{minipage}{3in}
\noindent
{\bf Sincerely,}

\vskip 2em

{\bf 
Jared Galloway, Bill Cresko, and Peter Ralph.
}\\
\end{minipage}

\vskip 4em

\pagebreak
\setcounter{page}{1}

%%%%%%%%%%%%%%
\reviewersection{AE}

\begin{quote}
I would ask that you add some text to the discussion addressing reviewer points 1.1 and 3.2.
\end{quote}


Here is Reviewer 1 point 1:
\begin{quote}
    \textit{``As expected, the genetic basis of the freshwater phenotype seems to simplify
    as migration increases -- higher rates of migration allow adaptive alleles of
    larger effect to travel more efficiently through the population, even though
    they are deleterious in the ocean.''}
    Very cool. It gives a very intuitive guideline as to which genetic
    architecture of local adaptation to expect given a certain migration rate.
    There is one neat aspect of this result that is perhaps worth better
    emphasizing: detectable $F_{ST}$ peaks at truly adaptive
    alleles can only be very few. If we had replicate runs we could say how many
    true $F_{ST}$ peaks, on average, we see at $M=1$ and $M=10$. Many published studies
    (particularly in marine animals) report dozens if not hundreds of $F_{ST}$ outliers
    throughout the genome; your simulation here suggests that most of them have to
    be false positives of some kind, especially since migration in the sea is
    likely to be high.
\end{quote}

% our previous \reply{
%     We agree with the reviewer that this is an important point,
%     and that our results are consistant with the above hypothesis. 
%     However, to quantify and assert that previous empirical studies 
%     are reporting false-positives would require many 
%     more simulations and analysis in many more situations,
%     which we believe is outside the realm of this paper.
% }

We have added, at \llname{disco11}:

\textit{
We find that, unsurprisingly, sharing of alleles across lakes leads to
a very simple genetic basis -- only tens of loci, at $M \ge 1$.
This suggests that all else being equal, systems with higher migration
should have fewer loci responsible for local adaptation.
However, ``all else is equal'' is rarely plausible:
the actual number of locally adaptive alleles in a particular system
depends on many other factors, including
the strength of selection, degree of pleiotropy, and the genetic basis of the trait.
A different genetic basis could well lead to a much larger number of alleles, even at high migration rates:
but, could these be reliably identified?
Next, we look at how well adaptive alleles are identified with genome scans.
}


% \textit{
% Because the detectable traits at high dispersal were very few,
% empirical studies in the situation of high dispersal which report
% hundreds or thousands of causal loci may benefit from exploring
% false-positive rates in similar modeling studies.
% }

\bigskip

Here is Reviewer point 3.2:
\begin{quote}
If possible, it would be interesting to check what type of mutations - in terms of effect size and in terms of dominance - fall under the 'captured', 'marine' and 'de novo' categories. I could well imagine, for example, that mutations that can stay in the marine populations relatively long are more likely to be recessive. Is this true?
\end{quote}

% our previous \reply{
%     We thought about this also, and did this comparison in some earlier simulations,
%     but any differential representation was not obvious
%     -- there was no large effect --
%     and that we'd need many more simulations to detect anything.
%     So: this seems plausible, but speculatively,
%     it doesn't seem to be a major driver of genetic architecture,
%     but rather a minor adjustment.
% }



We have added, at \llname{disco32}:

\textit{
We explored the possibility of dominance being a good predictor of
how long an adaptive allele can be maintained in the marine environment,
but found no effect strong enough to be seen with our relatively small sample size of adaptive alleles.
}
