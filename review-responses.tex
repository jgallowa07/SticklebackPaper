%%%%%%
%%
%%  Don't reorder the reviewer points; that'll mess up the automatic referencing!
%%
%%%%%

\begin{minipage}[b]{2.5in}
  Resubmission Cover Letter \\
    {\it G3}
\end{minipage}
\hfill
\begin{minipage}[b]{2.5in}
    Jared Galloway, \\
    William Cresko, \\
    \emph{and} Peter Ralph \\
  \today
\end{minipage}
 
\vskip 2em
 
\noindent
{\bf To the Editor(s) -- }
 
\vskip 1em

We are writing to submit a revised version of our manuscript, 
``A few stickleback suffice for the transport of alleles to new lakes''.
Many thanks to the reviewers for thoughtful and helpful comments,
to which we have responded in detail.
We sympathize that additional simulations would be informative and interesting,
but the amount of work that these would entail is beyond our capability at the moment.
We appreciate being explicitly given the option to omit them.
Otherwise, we think that the revisions have helped to clarify and generally improve the manuscript.
We hope you agree.

The responses to reviews includes page and line numbers that refer to the resubmitted manuscript;
we are also uploading a color-coded pdf that shows differences from the previously submitted version.


\vspace{5em}

\noindent \hspace{4em}
\begin{minipage}{3in}
\noindent
{\bf Sincerely,}

\vskip 2em

{\bf 
Jared Galloway, Bill Cresko, and Peter Ralph.
}\\
\end{minipage}

\vskip 4em

\pagebreak
\setcounter{page}{1}

%%%%%%%%%%%%%%
\reviewersection{AE}

\begin{quote}
    I agree particularly with questions from two of the reviewers asking for
    explanation or additional simulations on the founder population size, with
    reviewer 3's point about clarifying old vs new, and reviewer 1's request
    for more simulations. I do understand that simulations can be time
    consuming, however, and want to make clear additional simulations are not a
    requirement for acceptance (but would be rad if feasible). 
\end{quote}

Additional simulations exploring more of parameter space would be totally awesome, we agree.
However, besides the constraints on our time noted above,
we also note that it's not clear how much additional value it would have to delimit
the region of paramter space on which our results hold:
it would be more satisfying to develop theory that would apply to
larger populations of varying sizes with fluctuating migration rates, for instance.

\begin{point}{\revref}
    My lab and I also spent some time debating what was meant by the explanation at the
    bottom of page 13 -- I think more effort to clarify here would be helpful.
\end{point}

\reply{
    Thanks! We have rewritten this to hopefully be more precise. \revref
}

\begin{point}{}
    Finally, we also wondered why the simulations started at a trait value of 0
    instead of 10 -- it makes comparison of old vs new a bit harder than seems
    necessary. 
\end{point}

\reply{
    We don't have a good reason. Apologies.
}




%%%%%%%%%%%%%%
\reviewersection{1}


\begin{quote}
This is a very well done work addressing the fundamental problem of local
adaptation in metapopulations; stickleback are a nice study system but the
findings are relevant for a wide variety of systems. I enjoyed it very much,
and my only wish is that the simulation has been run more than once. This is
not necessary for speed of adaptation - within-simulation replication of lakes
is totally sufficient for this - but for aspects related to genome scanning,
numbers of adaptive alleles in a metapopulation, and haplotype-wise effect
sizes (table 1), since (especially at high M) the number of adaptive alleles is
very few across the whole metapopulation. That said, I do not insist on
rerunning the simulation, since it must be very computationally intensive.
\end{quote}

Thanks for the nice words!
We sympathize, but think that the number of simulations is sufficient
for at least estimating the rough size of these effects -- 
these effects are mostly monotonic in migration rate,
so that for instance we have an estimate of the number of adaptive alleles at $M=1.0$
but also know it is bracketed by the values at $M=0.1$ and $M=10$.
Further simulations would only give us more precise estimates of the mean
and an idea of the standard deviation.


\begin{point}{\revref}
    \textit{``As expected, the genetic basis of the freshwater phenotype seems to simplify
    as migration increases - higher rates of migration allow adaptive alleles of
    larger effect to travel more efficiently through the population, even though
    they are deleterious in the ocean.''}
    Very cool. It gives a very intuitive guideline as to which genetic
    architecture of local adaptation to expect given a certain migration rate.
    There is one neat aspect of this result that is perhaps worth better
    emphasizing: detectable Fst peaks at truly adaptive
    alleles can only be very few. If we had replicate runs we could say how many
    true Fst peaks, on average, we see at M=1 and M=10, Many published studies
    (particularly in marine animals) report dozens if not hundreds of Fst outliers
    throughout the genome; your simulation here suggests that most of them have to
    be false positives of some kind, especially since migration in the sea is
    likely to be high.
\end{point}

\reply{
    We agree with the reviewer that this is an important point,
    and that our results are consistant with the above hypothesis. 
    However, to quantify and assert that previous empirical studies 
    are reporting false-positives would require many 
    more simulations and analysis in many more situations,
    which we believe is outside the realm of this paper.
}


\begin{point}{}
Also, about Fig. S1: I am confused what is ``statistical power'' here :
``likelihood that a SNP will be predicted to have an effect on phenotype when
there is an effect to be detected''. In this model every SNP has at least *some*
effect on phenotype, so what defines ``effect to be detected'' then?
\end{point}

\reply{
    Actually, there are many SNPs that do not affect the phenotype,
    but these were added as a secondary step after simulation,
    so this is an easy mistake to make.
    We've added a reminder about this. \revref
}

\begin{point}{\revref}
    \textit{``Surprisingly, for all cases with the exception of M = 0.01, we found that the
    majority of adaptive allele origins traced back the original generation of
    inhabitants in the lake (Figure 7)''}
    This is really cool and counter-intuitive,
    but maybe it is not too surprising since first-arriving (``captured'') alleles
    are simply much more numerous relative to the alleles that trickle in later,
    *before the adaptation is complete*? Indeed, initial colonization involves
    arrival of 200 individuals in a single generation, which is equivalent to 200/M
    generations of subsequent immigration. Considering that at any appreciable M
    ($>=0.1$) adaptation happens very rapidly (fig 3 and S3), there might simply be
    not enough additional immigrant alleles to make a contribution. I wonder how
    this would work if colonization of a lake was a more gradual process, say 2
    individuals per generation (not all 200 at once)? Would the early-arriving
    alleles still have advantage? (I would bet yes)
\end{point}

\reply{
We agree that this is an important point to explore, and a good observation.
We have appended to the section on migration to more clearly address this \llname{disc_colon}.  
The available code could certainly be used in further studies to investigate this question more in-depth, but again feel that is outside the scope of this paper and have decidecided not to conduct more simulation and analysis.
}


%%%%%%%%%%%%%%
\reviewersection{2}

\begin{quote}
    This is an interesting, and as far as I can judge, a sound simulation study
    focused on population genetics of adaptation in the threespine stickleback.
    While the problem(s) approached in the manuscript are interesting and relevant,
    the way the ms has been pitched makes is mostly interesting for community of
    researchers interested on sticklebacks. By angling the story in more general
    terms and bringing stickleback in as model system, broader general appeal might
    have been achieved.
\end{quote}


\begin{point}{}
Latin binomial of the focal model species appears only in the references. It would make sense to incorporate it in the abstract and introduction given that the entire ms is focused on this species.
\end{point}

\reply{
    Good point; we've added it to the first use of the name after the abstract \revref.
}

\begin{point}{Title:}
I would make it clear that this is about the ``threespine stickleback'' - there are many different stickleback species.
\end{point}

\reply{
    We believe the first sentence in the abstract makes this clear to the reader.
}

\begin{point}{\revref}
``to millions of independently derived''. Do the references you cite say so? Can you really back up your claim of millions of freshwater populations?
\end{point}

\reply{
    This order of magnitude estimate is roughly based on the length of the coastline
    (roughly 10,000 miles) and the high density of freshwater lakes in many flat, coastal areas.
    We have reworded this to be more conservative, however. \revref
}

\begin{point}{\revref}
``heterogenous nature of divergence''. I think you should cite here the first (paper using global sampling) showing evidence for heterogenous nature of this genetic divergence: DeFaveri et al. (2011). Global analysis of genes involved in freshwater adaptation in threespine sticklebacks (Gasterosteus aculeatus). Evolution, 65(6), 1800-1807.
\end{point}

\reply{
    Thank you for pointing this out. The citations has been added.
}

\begin{point}{\revref}
P2. ``stickleback in them'' - threespine stickleback in them
\end{point}

\reply{
    Fixed.
}

\begin{point}{p.3 Methods}
From here on, the ms is generally very thinly referenced. In particular, it remains unclear to reader where the many equations come from: are they something authors have derived themselves (I guess not?), or who did that?
\end{point}

\reply{
    We're not sure what is meant here -- the equations in the Methods are almost entirely choices
    we made in setting up the model:
    for instance, the choice of a Gaussian fitness function is almost ubiquitous,
    and the equation for the phenotype in terms of the selection and dominance coefficients
    is just writing down the definition.
    Most of these are options available in SLiM, which we cite.
    We have included a reference for the (simple) defintion of $F_{ST}$ that we use \llname{Fst},
    and added a clarifying comment to the section of the Results and Discussion
    that is most equation-heavy \llname{Gil_cite}.
}

\begin{point}{p.5}
One is left to wonder if it realistic to think that the marine population is as big as 25 freshwater populations. Admittedly, this is discussed to some extent in the discussion, but given the abundance of the threespine stickleback in many seas (there is actually quantitative estimates of this in literature), it seems plausible that the marine population(s) is orders of magnitudes larger than that of typical freshwater population, even if combined. I guess a more quantitative view on this in light of literature would be something worth attempting.
\end{point}

\reply{
    We agree, this would be very interesting, and an informative exercise.
    However, we think it's outside the scope of this paper,
    and hope the reviewer agrees with our discussion of this point at \revref.
}

\begin{point}{\revref}
    Where does the mutation rate $10^{-10}$ come from? Why this particular value, and how are the results influenced by choice of this?
\end{point}

\reply{
    We chose this to reflect our guess at what might be a general consensus for the mutational target size
    of a polygenic trait (although we recognize that opinions differ).
    We've added some more discussion to contextualize our choice to the paper \revref.
    Some more on this: since per-bp mutation rates are likely closer to $10^{-8}$,
    a rate of $10^{-10}$ represents roughly one adaptive target per 100 bases
    across the $10^6$ bases where mutations that affect the phenotype might occur.
    This translates to a total \emph{potential} target size of $10^4$ bp,
    which seems generous but not impossible for a polygenic trait.
}

\begin{point}{}
P4. It is bit unclear here how LD as was dealt with in the simulations. Are all the loci assumed to be unlinked?
\end{point}

\reply{
    All chromsomes are linear with a per-nucleotide recombination rate of $10^{-8}$. This has been clarified here \revref.
}

\begin{point}{\revref}
What happens if generations are overlapping, as is the case in sticklebacks?
\end{point}

\reply{
    We know of no reason why in this case overlapping generation would affect the results.
}

\begin{point}{\revref}
Here the authors say that a different model in SLiM would have been more appropriate to be used, but that they ``do not expect this to influence the results''. This does not sound very good - why not to check this? Why not to rerun the analyses if a more appropriate model is available?
\end{point}

\reply{
In a perfect world, we would re-run and compare. 
However, to re-write this model as non-WF would require a relatively significant amount of change to the code as well as much more computational power to re-run the analysis.
We know of no reason to expect differences between the two implementations --
the main difference being in elegance --
and so have opted to stick with the current model. 
We have altered the wording a bit to better convey this \llname{realistic}.
}

\begin{point}{Fig 1}
The modeled scenario is probably correct over shorter geographic distances. But there is plenty of empirical evidence to show that marine threespine stickleback populations over larger geographic distances are not genetically homogenous. I think it would be appropriate to acknowledge this somewhere in manuscript.
\end{point}

\reply{
We agree, it should be made clear that the oceanic population is not panmictic but rather has one-dimential continuous spatial structure.
We have clarified this here \llname{continuous}.
}

\begin{point}{\revref}
``The efficacy of genomic scans of selection depends on gene flow''. I found this section rather uninformative: all this has been studied in quite detail already in many theoretical (and empirical) papers. This could be shortened. And references to earlier findings would be in place here too - there is no single reference to earlier works here at all.
\end{point}

\reply{
We believe this finding supports existing studies, specifically, how stronger genetic drift with less migration leads to higher background $F_{ST}$.
However, for our study, showing how greater sharing of adaptive alleles provides a shared signal across populations seemed necessary for a study of selection on standing variants.
Thus, we have added some more citations to previous work,
but otherwise have left this section as is.
}

\begin{point}{\revref}
``Simulation results align with..'' It is nice that the results are referenced back to theory. I was left to think that it might have been better to do this by combining this section with the actual results. Now the structure of the ms appears to be: 1) Results \& Discussion, 2) Discussion. Furthermore, references to theory are given in the first paragraph, but not after this. Readers do not know where all the equations are derived from.
\end{point}

\reply{
    We considered this sort of re-ordering when writing the paper,
    but decided that it would be more confusing to jump back and forth
    to the population genetics heuristics.
    The equations are standard, and we've tried better to flag this at the start of the section.
    \llname{Gil_cite}
}

\begin{point}{\revref}
``we found that standing..'' Is ``found'' right word here? You made an inference, even a guess, that this was so. But you did not prove this?
\end{point}

\reply{
    It is a good point that although we found linkage between opposite-effect alleles,
    this masking might not have a substantial effect.
    We've hedged a bit more in this statement.
}

\begin{point}{\revref}
    ``Population size''. Here it is argued that population sizes of marine and
    freshwater populations are not so different. I found it hard to believe this in
    the light of quantitative data there is. Could it simply be so that marine
    population is actually very much larger, but structured (as there is genetic
    evidence for this in many papers) so that the fraction feeding the freshwater
    colonizations is to some degree isolated from ones feeding the freshwaters in a
    particular area? This would in effect make relevant portion of the marine
    population smaller, without need to postulate that the total marine population
    equals in size to freshwater populations, and that there are millions of lakes
    inhabited by three spine sticklebacks. Note also that in introduction it was
    claimed millions of freshwater populations, here you say ``if not millions'' -
    inconsistent. 
\end{point}

\reply{
    We've made the two sections consistent,
    and still heartily agree that quantifying this would be an interesting and useful endeavor,
    but beyond the scope of the current paper.
    The proposed substructure in the ocean doesn't seem very plausible to us,
    although there is little enough data on oceanic demography that this may well be correct.
    Still, is our argument that they face similar constraints on breeding ground not convincing?
}




%%%%%%%%%%%%%%
\reviewersection{3}

\begin{quote}
    \ldots
    Overall, the paper finds clear support for the transporter hypothesis, and
    uncovers some key conditions under which the transporter mechanism is most
    likely to occur. An interesting insight is that little gene flow can suffice to
    provide enough standing variation in the marine gene pool for parallel
    freshwater adaptation to mostly involve pre-existing (instead of de novo). I
    also found interesting that freshwater-adaptive alleles can be maintained for a
    relatively long time in the marine population, thus indicating that it's not a
    necessity for gene flow to occurs at all times between freshwater and the sea
    for genetic variation to be shared by descent among freshwater populations. I
    also liked the paper's insights on the importance of considering entire
    adaptive haplotypes - and not only the number of adaptive alleles - for
    repeated adaptation.  
\end{quote}

\begin{quote}
    The paper is well written and I like this work and don't see any
    fundamental flaws in the way the simulations were set up or in how the results
    are interpreted. Of course, one always needs to make certain assumptions when
    setting up simulations (such as on effect sizes of loci etc.), but I think that
    the authors explore the parameter space appropriately. 
\end{quote}

Thanks! We're glad you liked it.

\begin{quote}
    In what follows, I provide suggestions on how to further improve the work. Most
    of these points should be understood as recommendations. A side-note:
    Please make sure to add line numbers to your next submission. It is
    difficult for reviewers to refer to specific text passages without line
    numbers being present.
\end{quote}

Sorry about that; we're submitting with line numbers this time.

\begin{point}{}
Founder effects followed by population growth: I wonder how founder effects during the colonization of the 'new lakes', followed by population growth, would influence the results? It seems that this was never explored, as each new lake had a constant population size of 200 individuals (rather low). I don't expect there to be a qualitative change in the overall results, but the relative contribution of de novo vs. standing variants to adaptation might well be influenced by a more colonization scenario. If the authors decided not to run additional simulations on this issue, it might at least be worth speculating on how founder effects followed by population growth might influence some of their results (such as the results shown in Fig. 7).
\end{point}

\reply{
Good point: we agree this would be one of the first additional parameters to vary.
We have discussed this further \llname{disc_colon},
but as we explained above, haven't done more simulations to expand coverage of parameter space.
}

\begin{point}{}
If possible, it would be interesting to check what type of mutations - in terms of effect size and in terms of dominance - fall under the 'captured', 'marine' and 'de novo' categories. I could well imagine, for example, that mutations that can stay in the marine populations relatively long are more likely to be recessive. Is this true?
\end{point}

\reply{
    We thought about this also, and did this comparison in some earlier simulations,
    but any differential representation was not obvious
    -- there was no large effect --
    and that we'd need many more simulations to detect anything.
    So: this seems plausible, but speculatively,
    it doesn't seem to be a major driver of genetic architecture,
    but rather a minor adjustment.
}

\begin{point}{}
I'm a bit confused by how the adaptive loci along the chromosome were modeled. I don't fully understand what you mean by mutations occurring at a rate $10^{-10}$ per locus per generations in ten regions of $10^5$ loci each (Page 4, 'Genetic architecture of the trait'). Why are the adaptive loci themselves organized in distinct regions? Why isn't there simply a given number of adaptive loci that are evenly spread across the chromosome?
\end{point}

\reply{
    We agree this was confusing and have edited this section to hopefully make it more clear \revreffull{2}{8}. 
    The motivation for 10 regions of effect was to roughly model the distribution of QTL seen in experiments,
    and to allow us to contrast closely linked variants (within the same region)
    with more distant ones.
}

\begin{point}{Results \& Discussion section}
    Sometimes, it is not fully clear whether you talk about the 'old' lakes or the 'new' lakes. This is often unclear, for example, in the section ``Rapid local adaptation at intermediate gene flow rates''. Could you make this clear throughout the manuscript?
\end{point}

\reply{
Good point, we have clarified a few ambiguous spots, e.g., \revref.
}

\begin{point}{Figure 2}
    Are these the results after 50k generations?
\end{point}

\reply{
    Actually, at the time of introduction (so 100,000 generations), thank you for the clarification, this has been noted in the figure caption. 
}

\begin{point}{}
    In the legend of Figure 4, it might be good to re-iterate what the number on top of the panels means exactly. Say that these are actual numbers of migrants per population and what the population size is. What is meant by 'M' is said more clearly in the legend to Figure 6, yet, it would be good to here also state (again) how large the populations are. M can only be understood in context of the actual population size. Otherwise, you would need to provide 'm'.
\end{point}

\reply{
    We have revised the figure caption and added the label $M$ to the header.
}

\begin{point}{}
Figure 5: delete '100' along the x-axis.
\end{point}

\reply{
Good point: Fixed.
}

\begin{point}{}
Typo on page 14: two '- -' instead of only one.
\end{point}

\reply{
    Whoops! Fixed; thanks. \revref
}

\begin{point}{}
Page 2, 2nd paragraph (beginning): When you talk about the evolutionary history of stickleback, an appropriate citation would also be Bell \& Foster 1994.
\end{point}

\reply{
Good catch - added. \revref
}

\begin{point}{}
Page 2, 2nd paragraph: You provide several empirical papers that demonstrate the re-use of standing variation in stickleback. Your list of papers is fair, although I don't think you chose the best papers of all the ones available. There are papers that are not cited that directly show evidence of standing variation in the sea that is being re-used for freshwater adaptation, such as: Jones et al. 2012 (Curr Biol); Haenel et al. 2019 (EvolLet), Feulner et al. 2014 (MolEcol) etc. Evidence for the reuse of the exact same genetic variation at different locations on the northern hemisphere also comes from inversions in the stickleback genome (Jones et al. 2012 (Nature); Roesti et al. 2015 (NatCom)).
\end{point}

\reply{
Great catch, citations added. \revref
}

\begin{point}{\revref}
Page 2, 2nd paragraph: you here talk about 'similar phenotypes' that have evolved repeatedly in freshwater, but you provide only two examples. The cited papers are mostly concerned with the genetic basis of a specific phenotype that has repeatedly evolved (plate loss). However, at the phenotypic level, there are many more examples for marine-freshwater adaptations in stickleback.
\end{point}

\reply{
Good point; we have added citations to more examples. \revref
}

\begin{point}{}
Page 3, 2nd sentence ('An alternative hypothesis ... maintained in the marine .... due to recurrent gene flow'). I would replace 'maintain' with 'present' because it is unclear at this point whether alleles are actually maintained in the marine population (i.e., stay there for many generations), or whether they are just 'washed into' the marine population over and over again from freshwater without staying long in the marine population at all.
\end{point}

\reply{
Good point. Changed. \revref
}


\begin{point}{}
It might be good to cite Hohenlohe et al. 2010 and/or Roesti et al. 2014 to justify that the marine population is modeled to be panmictic (page 4, top). Both these studies have shown very little genome-wide differentiation between samples taken at different marine locations.
\end{point}

\reply{
This was clarified \llname{continuous}. The marine population was not modeled as panmictic but rather a unidimensional continuous space with dispersal.

